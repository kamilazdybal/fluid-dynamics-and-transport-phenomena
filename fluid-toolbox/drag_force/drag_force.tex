The mathematical description of the drag force begins with making a guess. It is an intuitive guess which answers the question: what physical quantities affect the value of the drag force on an object moving through a fluid? The thought process done on this question gives the following four quantities:

relative fluid-object velocity $\upsilon$ \,\,\,\,\,\,\,\,\,\,\,\,\,\,\,  fluid density $\rho$ \,\,\,\,\,\,\,\,\,\,\,\,\,\,\, fluid viscosity $\mu$ \,\,\,\,\,\,\,\,\,\,\,\,\,\,\, geometry of an object $D$

We assume that the drag force is a combination of these quantities to some yet unknown powers and we use dimensional analysis to find those powers.

\begin{equation}
F_D = \upsilon^a \rho^b \mu^c D^d
\label{eq:drag_force}
\end{equation}

Writing out the units of the above equation we get:

\begin{equation*}
\Big[ \frac{kg \cdot m}{s^2} \Big] = \Big[ \frac{m}{s} \Big]^a \cdot \Big[ \frac{kg}{m^3} \Big]^b \cdot \Big[ \frac{kg}{m \cdot s} \Big]^c \cdot \Big[ m \Big]^d
\end{equation*}

Shuffling around a bit:

\begin{equation*}
kg \cdot m \cdot s^{-2} = kg^{b+c} \cdot m^{a -3b - c + d} \cdot s^{-a - c}
\end{equation*}

Hence:

$1 = b+c$

$1 = a - 3b - c + d$

$-2 = - a - c$

The simple fact that this set of equations cannot be solved exactly (there is four  uknown powers and only three equations) is going to call experiment for help, which you will notice in the next few passages.

Let's then write the set of equations by means of one of the unknowns $c$:

$a = 2 - c$

$b = 1 - c$

$d = 2 - c$

\newpage

Subsituting back to the equation \ref{eq:drag_force} we get:

\begin{equation}
F_D = \upsilon^{2 - c} \rho^{1 - c} \mu^c D^{2 - c} 
\label{eq:drag_force_powers}
\end{equation}

Structuring the equation still a bit gives:

\begin{equation}
F_D = \upsilon^2 \rho D^2 \cdot \Big[ \frac{\upsilon \rho D}{\mu} \Big]^{-c}
\label{eq:drag_force_powers}
\end{equation}

It always feels comfortable to find the Reynolds number in your equation, so there it is:

\begin{equation}
F_D = \upsilon^2 \rho D^2 Re^{-c}
\label{eq:drag_force_powers}
\end{equation}

We get finally that the drag force is proportional to some unknown power of the Reynolds number.


In fact, we will not leave it there yet, since there is an interesting last point to say. Instead of the above equation, we will say that the drag force is proportional to some unknown function $C_D$ of the Reynolds number. We also recognise that we may rewrite the quantity $\upsilon^2 \rho$ as the dynamic pressure $\frac{\upsilon^2 \rho}{2}$, since multiplying the right hand side by $\frac{1}{2}$ will not spoil the dimensional equality of both sides. The quantity $D^2$ has got the unit of area, so we exchange it for the quantity $A_{\perp}$, representing the frontal area of an object. The equation for the drag force becomes:

\begin{equation}
F_D = \frac{\upsilon^2 \rho}{2} A_{\perp} C_D (Re)
\label{eq:drag_force_powers}
\end{equation}

The unknown function $C_D (Re)$ is called the \textbf{drag coefficient}. That is where we call for experiment.

Questions:

What would have happened if we wrote the powers in terms of some power other than $c$?








